\chapter*{卷首语}

\noindent
\heading{从能跑到健壮的转变}

你好,未来的Python工程大师!

如果你已经编写Python代码一两年,能独立完成功能,解决日常编程任务,那么你已告别编程小白阶段,是一位合格的编码者。但或许你也遇到新困惑:为何代码在本地运行正常,到生产环境就出错?为何项目运行一段时间后维护艰难?为何程序实现了功能,性能却远低于预期?

这恰恰说明,从能写出可运行代码的编码者,进化为能交付健壮、高效、可维护系统的专业Python工程师,中间有巨大的工程可靠性能力鸿沟。

本书正是为此而写。它并非讲解Python基础语法的入门书,而是面向有一定基础的开发者,系统性提升工程化能力的进阶指南。全书围绕四大支柱展开:从环境与依赖管理的确定性开始,到基于契约的类型系统与质量审计,再到并发模型与性能洞察,最后是测试、配置与可观测性的全链路保障。

我们的目标只有一个:让你迅速从``会写代码''的开发者,成长为能独立设计、交付高可靠项目的工程指挥官。让我们携手,将你手中的Python代码,升级为真正的工业级工程实践!

\noindent
\heading{写作动机}

近年来,我在指导研究生与带领算法团队进行算法研发时,经常观察到一个现象:许多具备不错算法功底的工程师,能够编写功能性代码,但在将模型封装为服务、切入业务逻辑并构建可维护系统时,却常陷入开始能跑,越维护越难的困境。

这使我意识到,工程化能力的缺失并非个例。因此,我将分散的经验系统整理,形成了这本面向“非小白”的工程化进阶指南——不是教你写代码,而是讨论如何让代码在复杂环境中长期稳定运行。

\noindent
\heading{AI时代的思考:当AI能写代码,工程化能力更稀缺还是过时?}

AI助手日益普及,那么,都可以用AI写代码了,掌握工程化能力还有必要吗?

我把这个问题抛给了AI本身。大模型的回答很明确:“这本书的内容,在AI时代不仅没有过时,反而变得更加重要、更加核心。”

这正是我想告诉你的:AI越是能生成代码,工程化能力就越显珍贵。AI降低了编码门槛,却抬高了工程门槛。它能快速产出能跑的代码,却难以理解系统复杂性、预判长期维护成本、权衡架构取舍。你花3小时写的代码,AI 30秒就可以生成,但你的系统在线上崩溃时,AI只会沉默。

可见,在AI时代,优秀工程师的核心价值已不再是编码速度,而是工程审美——对系统健壮性、可维护性和扩展性的深度判断。这正是本书希望赋予你的能力:不是与AI竞争写代码,而是驾驭AI,让它生成的代码成为工业级系统的一部分。

\noindent
\heading{名称大小写说明}

本书涉及大量开源工具,其名称大小写均遵循官方格式,如asyncio、HTTPX、uv、Loguru、structlog等,正文中亦保持此规范。

\noindent
\heading{读者群体}

本书适合以下读者:

\begin{itemize}
    \item 已有一两年Python开发经验,希望系统提升工程化能力的开发者;
    \item 从事数据分析、机器学习等方向,希望将模型工程化支持业务需求的工程师;
    \item 想从``AI使用者''变为``AI驾驭者'',将成代码转化为可靠产品的团队;
    \item 准备从``写脚本''转向``做系统'',在AI时代保持竞争力的Python爱好者。
\end{itemize}


如果你正站在从``会写代码''到``能构建可靠系统''的十字路口,这本书正是为你而写。

愿你在此书中,找到属于自己的工程之路,培养超越AI的工程审美。

\begin{flushright}
夏天~~~~ 

中国人民大学
\end{flushright}

