\chapter*{卷首语}

\noindent
\heading{从能跑到健壮的转变}

你好,未来的Python工程大师!

如果你已经编写Python代码一两年,能独立完成功能,解决日常编程任务,那么你已告别编程小白阶段,是合格的编码者。

但或许你也遇到新困惑:为何代码在本地运行正常,到生产环境就出错?为何项目运行一段时间后维护艰难?为何程序实现了功能,性能却远低于预期?

这是因为,从能写出可运行代码的编码者,进化为能交付健壮、高效、可维护系统的 Python 工程师,中间有巨大鸿沟,即工程化能力。

本书正是为你打造。它并非讲解Python基础语法的入门书,而是面向有一定语言基础的开发者,系统性、实践性提升工程化水平的进阶指南。

在本书中,我们将共同探索以下问题:

\begin{itemize}
    \item {环境的统一与加速}:彻底摆脱pip install的混乱,掌握虚拟环境、依赖锁定与自定义源加速的专业技巧。
    \item {质量与健壮性}:运用静态类型系统、自动化代码规范与设计模式,确保代码的健壮性与可维护性。
    \item {并发与性能优化}:了解GIL机制,掌握多进程/多线程与asyncio异步编程,并通过 Profiling 和 Cython 突破性能瓶颈。
    \item {测试、配置与可观测性}:构建专业的单元测试与集成测试体系,掌握生产级配置管理,实现系统的结构化日志与高效调试。
\end{itemize}

我们的目标只有一个:让你迅速从``会写代码''的数据分析工程师或AI算法工程师,成长为能独立设计、管理和交付高质量、高可靠性Python项目的高级专业工程师。

让我们携手,将你手中的Python代码,升级为真正的工业级工程实践!

\vspace{1em}

\noindent
\heading{写作动机}

在指导研究生的过程中,以及近年来带领算法工程师团队进行专业领域大模型研发时,我观察到一种现象:许多刚毕业的研究生,甚至已有两三年工作经验的算法工程师和数据分析师,虽然能够编写出功能性的Python代码,但在工程化方面却依然存在明显短板,常常陷入“开始时代码能运行,但越往后维护越困难,效率也越来越低”的困境。

你可能拥有深厚的机器学习或深度学习背景,能够使用Python编写模型训练脚本,甚至能够阅读和修改深度学习框架的底层代码。然而,将模型封装成服务并在生产环境中稳定运行,还需要另一套技能——工程化能力。

为便于经验分享,我断断续续积累了部分文档,并在交流过程中意识到工程化能力的缺失并非个例,而是一个普遍存在的挑战。因此,我决定将这些分散的经验系统化,整理成一本专注于工程化实践的书籍,旨在帮助那些已经掌握Python基础但渴望提升工程能力的开发者。这便是本书的由来——一本面向``非小白''的Python工程化进阶指南。

\vspace{1em}

\noindent
\heading{名称大小写说明}

本书涉及大量开源项目,其名称的首字母大小写形式不完全一致。考虑到每个工具的大小写是其名称的一部分,在正文中应予以保留,因此书中保持了其官方格式,如asyncio、HTTPX、uv、Loguru、structlog等。

\vspace{1em}

\noindent
\heading{读者群体}

本书适合以下读者:

\begin{itemize}
    \item 已有1--3年Python开发经验,希望系统提升工程化能力的开发者;
    \item 从事数据分析、机器学习、AI算法等方向,希望将模型工程化的工程师;
    \item 希望构建可维护、高性能、高可靠Python项目的团队技术负责人;
    \item 准备从``写脚本''转向``做系统''的Python爱好者。
\end{itemize}

如果你正站在从初学已入门到专业要升级的十字路口,那么这本书,正是为你而写。

\vspace{1em}

愿你在此书中,找到属于自己的工程之路。

\begin{flushright}
夏天~~~~ 

中国人民大学
\end{flushright}
