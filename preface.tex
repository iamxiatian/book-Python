\chapter*{卷首语}

\section*{从能跑到健壮的转变}

你好,未来的 Python 工程大师!

如果你已经编写 Python 代码一两年,能独立完成功能,解决日常编程任务,那么你已告别``编程小白''阶段,是合格的编码者。

但或许你也遇到新困惑:为何代码在本地运行正常,到生产环境就出错?为何项目运行一段时间后维护艰难?为何程序实现了功能,性能却远低于预期?

这是因为,从能写出可运行代码的编码者,进化为能交付健壮、高效、可维护系统的Python工程师,中间有巨大鸿沟,即工程化能力。

本书正是为你打造。它并非讲解 Python 基础语法的入门书,而是面向中高级开发者,系统性、实践性提升工程化水平的进阶指南。

在本书中,我们将共同探索并攻克以下核心领域:

\begin{enumerate}
    \item 环境的统一与加速:彻底摆脱 \mintinline{bash}{pip install}的混乱,掌握虚拟环境、依赖锁定与自定义源加速的专业技巧。
    \item 质量与健壮性:运用静态类型系统、自动化测试、规范化配置与结构化日志,确保代码在任何环境都坚不可摧。
    \item 并发与性能优化:了解GIL机制,掌握多进程/多线程与asyncio异步编程,并通过 Profiling 和 Cython 突破性能瓶颈。
    \item 协作与交付:建立专业的 Git 协作流程,将 CI/CD 融入日常开发,最终通过 Docker 容器化和高可用架构,自信地将项目部署到云端。
\end{enumerate}

我们的目标只有一个:让你迅速从``会写代码''的数据分析工程师或者AI算法工程师,成长为能独立设计、管理和交付高质量、高可靠性 Python 项目的高级专业工程师。

让我们携手,将你手中的 Python 代码,升级为真正的工业级工程实践!

\section*{英文开源项目名称的大小写说明}

本书涉及了大量的开源项目,这些项目名称的首字母大小写形式不完全一致,笔者认为每个工具的大小写都是其名称的一部分,在正文中应予以保留,因此正文中保持了其官方格式,如asyncio、HTTPie、Git、uv等。
